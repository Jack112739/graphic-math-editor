%% 75, 14, , 
\label{eigenvectors definition}

a vector $v$ is called eigenvector of the matrix $M$ \\$\iff ~ \exists \lambda \in \mathbb{C} : Mv = \lambda v$ \\the number $\lambda$ is called eigenvalue of $M$ respected to $v$
​
%% 72, 787, , 
\label{same kernel lemma}
$S$ is a symmetric matrix\\$\implies $S$ \text{ and } S^n$ h​ave the same kernel for all $n$

\begin{proof}
%% 51, 315, , 
\label{input}

​a symmetrix matrix $S$
%% 257, 258, , , input
\label{lemma 2}
$\ker S^n \subset \ker S^{n+1}$ for all $n$​

\begin{proof}
%% 107, 265, , 
\label{#17}

if $v$ is a vector in $\ker S^n$​
%% 211, 614, , , #17
\label{#18}

​$S^{n+1}v = S(S^n v) = 0$
%% 328, 951, , , #18
\label{#19}

​QED

\end{proof}
%% 231, 603, , , input, lemma 2
\label{lemma 1}
$\ker S = \ker S^2$ ​

\begin{proof}
%% 84, 849, , 
\label{#0}

Asume $v$ is a vector in $\ker S^2$​
%% 224, 852, , , #0
\label{#2}

$0 = v^TS^2v = (Sv)^2 \implies Sv = 0$​
%% 98, 478, , , lemma 2
\label{#14}

​as in \ref{lemma 2} $\ker S \subset \ker S^2$
%% 247, 619, , , #2, #14
\label{#16}

QED​

\end{proof}
%% 379, 624, , , lemma 2, lemma 1
\label{#13}

$\ker S \subset \ker S^n \subset \ker S^{2^n} \equiv \ker S$​
%% 491, 935, , , #13
\label{#11}

QED​

\end{proof}
%% 76, 529, , 
\label{#1}

Let $S$ be a symmetrix real matrix
%% 246, 282, , , eigenvectors definition, #1
\label{v_i definition}

Let $\lambda_1, \lambda_2,...\lambda_n$ ​​be the eigenvalues of $S$
%% 234, 703, , , v_i definition, #1
\label{#6}

$S - \lambda_i I$ is a symmetrix matrix for all $i$​
%% 204, 1034, , , #6, same kernel lemma
\label{#7}

$\ker (S - \lambda_i I) \equiv \ker (S - \lambda_i I)^m$ for all $i, m$​
%% 416, 1091, , , #7
\label{conclusion}

​QED
