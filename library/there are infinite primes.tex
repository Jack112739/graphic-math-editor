%created on 4/30/2024, 11:28:57 PM:


%% definition of prime, 273, 16, 88, 279, 0, 0
\label{of prime}
a number $p$ is a prime if and only if it is greater than 1 and have exactly two positive divisor, which is 1 and itself
%% theorem, 136, 890, 87, 172, 0, 0
every number $N > 1$ can be factored into products of primes
%% , 54, 319, 46, 480, 409, 480
if there are finitely many prime then there will
be a number $N > 1$ that divide no primes
\begin{proof}
%% assume, 31, 301, 84, 147, 0, 0
the set of all primes $\mathcal P$ is finite
%% , 51, 21, 125, 190, 0, 0, #13
we can choose the number
$$N = \prod_{p \in \mathcal P} p + 1$$
%% , 240, 241, 50, 213, 0, 0, #15
$N \equiv 1 \mod p $ for all $p \in \mathcal P$
%% QED, 280, 72, 50, 65, 0, 0, of prime, #18

\end{proof}
%% contradiction, 375, 894, 50, 134, 0, 0, #9, #7
